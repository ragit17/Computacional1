\documentclass{article}

% set font encoding for PDFLaTeX or XeLaTeX
\usepackage{ifxetex}
\ifxetex
  \usepackage{fontspec}
\else
  \usepackage[T1]{fontenc}
  \usepackage[utf8]{inputenc}
  \usepackage{lmodern}
  \usepackage{graphicx}
\fi



% used in maketitle
\title{Actividad 1: La Atmósfera Terrestre}
\author{Rolando Abdel Fimbres Grijalva}
\date{25 de Enero, 2018}

% Enable SageTeX to run SageMath code right inside this LaTeX file.
% documentation: http://mirrors.ctan.org/macros/latex/contrib/sagetex/sagetexpackage.pdf
% \usepackage{sagetex}

\begin{document}
\maketitle{Atmósfera de la Tierra:}
La atmósfera terrestre es un conjunto de capas de gases, comunmente conocida como aire.Ésta rodea la Tierra y es retenida por su gravedad. La atmósfera proteje la vida terrestre pues crea presión permitiendo la existencia de agua líquida sobre la superficie, absorbe radiación solar ultravioleta, calienta la superficie terrestre mediante la retención de calor y reduce temperaturas extremas entre el día y la noche.

\begin{figure}[h!]
    \includegraphics[width=\linewidth]{m.JPG}
    \caption{La Luna}
\end{figure}

\section{Composición}
Los tres mayores gases que constituyen la atmósfera terrestre son: oxígeno, nitrógeno y argón. El vapor de agua representa aproximadamente un 0.25\%. La concentración de vapor de agua varía desde 10 ppm en las porciones más frías en la atmósfera hasta 5\% por volumen en calientes, masas húmedas de aire y concentraciones de otros gases atmosféricos típicamente llamados $aire$ $seco$.
El resto de los demás gases son usualmente referidos como $gases$ $rastro$, entre los cuales se encuentran los gases invernadero. Además también podemos encontrar contaminantes tales como lo son aerosoles, cloruros y sulfuros.

\section{Estructura Atmosférica}
En general, la presión y densidad decresen con el aumento de altitud en la atmósfera. Sin embargo, la temperatura cuenta con un perfil complicado con la altitud y puede mantenerse relativamente constante o incrementar en ciertas regiones. Las regiones de menor a mayor son las siguientes: tropósfera, estratósfera, mesósfera, termósfera y exósfera.

\begin{figure}[h!]
    \includegraphics[width=\linewidth]{eat.png}
    \caption{Atmósfera terrestre.}
\end{figure}

\subsection{Tropósfera}
Ésta se encuentra desde la superficie hasta los 12 km de altura. Esta está limitada por la tropopausa, una frontera marcada en la mayoría de los lugares por una inversión de temperatura. La tropósfera contiene aproximadamente un 80\% de la masa atmosférica terrestre. Ésto además la convierte en la capa más densa de todas. Cerca de todo el vapor atmosfério se concentra aquí, pues es la capa donde el clima terrestre toma lugar.
\subsection{Estratósfera}
La estratósfera se encuentra en una altura de 12 km hasta aproximadamente 55 km donde se encuentra la estratopausa. La presión atmosférica en la cima de la estratósfera es de 1/1000 la presión al nivel del mar. Contiene a la capa de Ozono la cual alberga relativamente altas concentraciones de ese gas. La estratósfera define una capa en la cual la temperatura incrementa al aumentar la altitud. Ésto provocado por la absorción de radiación ultravioleta del Sol por la capa de ozono.
\subsection{Mesósfera}
Es la tercera mayor capa de la atmósfera terrestre, extendiéndose desde los 50 hasta los 85 km. Las temperaturas bajan en direccion a la alta mesopausa constituyendo la capa en la mitad de la atmósfera terrestre y siendo el lugar más frío de la Tierra alcanzando los $-85^\circ$ C.
Justo debajo de la mesopausa el aire es tan frío que forma \textit{nubes mesosféricas polares}. Ésta capa también es donde más meteoros se desintegran al entrar en la atmósfera. Su altura es muy alta para aeronaves y muy baja para naves orbitantes, sin embargo es accedida por cohetes sonda y otras aeronaves propulsadas por cohetes.
\subsection{Termósfera}
Es la segunda capa más alta de la atmósfera terrestre y se extiende desde la mesopausa ubicada aproximadamente a 80 km hasta la termopausa que se ubica dentro de un rango de 500 a 1000 km. La temperatura aquí varía debido a cambios en la actividad solar. Debido a que la termopausa yace en la barrera inferior de la exósfera tambipen es conocida como exobase. En la parte baja de la termósfera desde los 80 hasta los 550 km encontramos la ionósfera.

La temperatura incrementa gradualmente con la altura a diferencia de la estratósfera que sufre una inversión de temperatura debido a la absorción de radiación por la capa de ozono. El aire se encuentra tan enrarecido que una molécula (de oxígeno) viaja a un promedio de un kilómetro entre colisiones con otras moléculas. Aunque la termósfera presente una alta proporción de moléculas con alta energía, no se sentiría calienta para un humano, pues su densidad es muy baja para conducir energía hacia o desde la piel.
La capa esta libre de nubes y vapor de agua sin embargo tiene lugar fenómenos no hidrometeorológicos como lo son las auroras boreales.
\subsection{Exósfera}
Es la capa más externa de la atmósfera terrestre y se extiende desde la exobase a 700 km hasta los 10,000 km combinandose con el viento solar.
Ésta capa se compone mayormente de extremadamente bajas densidades de hidrógeno, helio y algunas molpéculas pesadas como el nitrógeno, oxígeno o dióxido de carbono. Los átomos y moléculas están tan dispersos que viajan cientos de kilómetros sin colisional entre ellos. Aunque la exósfera no se comporta como un gas y sus partículas constantemente escapan hacia el espacio, éstas siguen una trayectoria balística y pueden migrar dentro y fuera de la magnetósfera o viento solar. La exósfera contiene la mayoría de los satélites orbitando la Tierra.

\section{Propiedades Físicas}
\subsection{Presión y Grosor}
La presión atmosférica promedio al nivel del mar es definida por la \textit{International Standard Atmosphere} es de 101325 pascales. También referida como unidad estándar de atmósferas (atm). La masa atmosférica total es de $5.1480 \times 10^{18}$, la presion atmosférica es el peso total de aire sobre unidad de área en el punto donde la presión fue medida.
Si la masa atmosférica tuviera una densidad uniforme desde el nivel del mar, terminaria abruptamente a los 8.5 km de altitud. En la realidad ésta decrece exponencialmente con la altitud, reduciendose a la mitad cada 5.6 km.
En resumen, la masa atmosférica terrestre se distribuye aproximadamente de la siguiente forma:
\begin{itemize}
\item 50\% está bajo los 5.6 km.
\item 90\% está bajo los 16 km.
\item 99.99997\% está debajo de los 100 km, la línea de Kármán.
\end{itemize}
\subsection{Temperatura y Velocidad del Sonido}
La temperatura decrece con la altitude comenzando al nivel del mar, pero, las variaciones comienzan sobre los 11 km donde la temperatura se estabiliza a través de una larga distancia entre el resto de la tropósfera. En la estratósfera, empezando sobre los 20 km, la temperatura incrementa con la altura, debido al calentamiento en la capa de ozono causado por la captura significativa de radiación ultravioleta del Sol. Otra region que incrementa su temperatura debido a la altura ocurre en la termósfera sobre los 90 km. Dado que un gas ideal de composición constante la velocidad del sonido depende meramente de la temperatura y no de la presión del gas o su densidad, la velocidad del sonido en la atmósfera toma un complejo comportamiento debido a la temperatura.
\subsection{Densidad y Masa}
La densidad del aire al nivel del mar es de aproximadamente $1.2 kg/m^3$. Ésta no es medida directamente, sino, calculada mediante mediciones de temperatura, presión y humedad utilizando la ecuación de estado de aire par aun gas ideal. La masa atmosférica promedio es de $5\times10^{15}$ toneladas o $1/1200000$ veces la masa de la Tierra.
\section{Propiedades Ópticas}
La radiación solar es la energía que la Tierra recibe del Sol. La Tierra tambipen emite una radiación de regreso al espacio, pero a longitudes de onda más largas que a simple vista no podemos percibir.
\subsection{Dispersión}
Cuando la luz atravieza la atmósfera terrestre los fotones interaciónan mediante la dispersión. Si la luz no interacciona con la atmósfera es llamada radiación directa y es lo que notas al mirar directamente al Sol. La radiación indirecta es aquella luz dispersada en la atmósfera. Un ejemplo es el llamado dispersión de Rayleigh, en éste longitudes de onda corta (azules) se dispersan más rápido que las largas (rojas). Ésta es la razón por la que el cielo se ve azul, en realidad vemos luz azul dispersada. También es por lo que las puestas de sol se ven rojas, pues al estar el Sol más cerca al horizonte, los rayos de Sol pasan a través de más atmósfera que lo normal para alcanzar tus ojos. Mucha de la luz azul ha sido dispersada, dejando a la luz roja en la puesta de sol.
\subsection{Absorción}
Distintas moléculas absorben diferentes longitudes de onda de radiación. Cuando una molécula absorbe a un fotón, incrementa la energía de la molécula. Ésto calienta la atmósfera.
\subsection{Emisión}
Los objetos tienden a emitir cantidades y longitudes de onda de radiación dependiendo de su curva de emisión de cuerpo negro. Por lo tanto objetos calientes emiten más radiación con cortas longitudes de onda mientras que objetos fríos emiten menos radiación con longitudes de onda más largas. Éste fenómeno ayuda a enfriar la superficie terrestre durante las noches. El efecto invernadero se relaciona de manera directa con los efectos de absorción y emisión.
\section{Circulación}
La circulación atmosférica es un movimiento de aire presentado a través de la tropósfera en el cual el calor se distribuye a lo largo de la Tierra. La estructura a gran escala de la circulación varía año con año. Pero su estructura básica se mantiene constante siendo determinada por la rotación terrestre y la diferencia de radiación solar entre el equador y los polos.

\begin{figure}[h!]
    \includegraphics[width=\linewidth]{atci.png}
    \caption{Circulación del aire.}
\end{figure}

\section{Bibliografía}

\begin{itemize}
\item Zimmer, C. (2013).\textit{Earth's Oxygen: A Mystery Easy to Take for Granted}. New York Times.
\item Russell, R. (2008). \textit{The Termosphere}.
\item Wallace, J. M. & Hobbs, P. V. (2006). \textit{Atmospheric Science: An Introductory Survey}.
\end{itemize}

\section{Apéndice}
1.-¿Qué fue lo que más te llamó la atencion de esta actividad?

El uso del lenguaje Tex para el desarrollo de documentos científicos que permite una visualización más clara de diversos conceptos.

2.-¿Qué fue lo que se te hizo menos interesante?

Citar en estilo APA, usualmente olvido como se hace a pesar de haberlo hecho muchas veces anteriormente.

3.-¿Qué cambios harías para mejorar esta actividad?

Me gustaría que se hubieran sorteado distintos temas a cada alumno para así conocer más acerca de varias temáticas.

4.-¿Cuál es tu primera impresión de uso de LATEX?

Pensé que sería muy complicado.

5.-¿El tiempo sugerido para esta actividad fue suficiente?

Si, lo fue.

6.-¿Encontraste algún documento o recurso en línea útil que quisieras compartir con los demás?

Si, hace tiempo encontre un pdf en línea llamado \textit{Edición de Textos Científicos Latex}, por Alexander Borbón y Walter Mora. Éste habla acerca de la coposición de textos, diseño editorial, gráficos, inkscape, presentaciones beamer; entre otras cosas, además cuenta con licencia creative commons.

\end{document}
